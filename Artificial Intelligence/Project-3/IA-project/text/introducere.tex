 \section{Introducere}

Pentru realizarea acestui proiect au fost nevoie de cunostințe legate de planificarea unei probleme și găsirea pașilor care duc la un scop bine stabilit.
Astfel am folosit limbajul PDDL, care generează toate stările unei probleme, ce are un domeniu unic.
\\ 

Componentele unei sarcini de planificare în PDDL sunt:
 \begin{itemize}
    \setlength\itemsep{0em}
    \item Objects = lucruri din lumea reală care ne interesează
    
    \item  Predicates = proprietăți ale obiectelor care ne interesează
    \item Initial state  = starea lumii în care începem
    \item Goal specification = lucruri care vrem să fie adevărate
    \item Actions/Operators =  modalități de schimbare a stării lumii
\end{itemize}

O problemă de planificare este creată prin asocierea unei descrieri de domeniu cu o descriere a problemei. Condițiile pre și post ale acțiunii sunt exprimate ca propoziții logice construite din predicate și termeni de argumente și conectivități logice.
\newline
\newline
Sarcinile de planificare specificate în PDDL sunt separate în două fișiere diferite:

 \begin{itemize}
    \setlength\itemsep{0em}
    \item Fișier de domeniu care conține predicatele și acțiunile
       \item Fișier cu probleme care conține obiectele, starea inițială și specificația goal-ului
    
\end{itemize}

Structura fișierului de domeniu este următoarea:
\bigskip
\\

(define (domain <domain  name> )

 (:predicates <predicate>)

         (:action  <prima actiune>)
      
         .......................
      
         (:action  <ultima actiune>)
)

\bigskip

Structura fișierului de probleme este următoarea:
\bigskip
\\

(define (problem <problem name>)

        (:domain <domain name>)
        
        (:objects  <obiecte>)
    
        (:init <starea initiala>)
     
        (:goal <starea finala>)
)