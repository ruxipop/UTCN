\section{Personal contribution}

% =============================================  a-b pruning in ABAgent ============================= 
\subsection{Question 11 - Define and solve your own problem.}
% enuntul intrebarii
In this section the solution for the following problem will be presented: \newline

%%%%% PLEASE WRITE THE PROBLEM DEFINITION HERE %%%%%%%

\textit{"In search.py, implement \textbf{IterativeDeepeningSearch(IDS) algorithm} to solve {\textbf{SearchProblem}} from searchAgents.py". }

\subsubsection{Code implementation}
This sub-section is dedicated to showcasing your own solution that you came up with for solving the above question. One has to put here any \textbf{code} that has been used for solving the above task, along with \textbf{comments} that explain every design decision made. To reference the code, please make use of the \textit{code lines number}. Additionally, complete this sub-section with any \textbf{command configurations} that you may have used during the implementation or testing process (please fill in \textit{just the arguments}). \\

\textbf{Code:}

% a se completa fisierul code/11_prop_problem.py
\inputminted[linenos]{python}{code/11_prop_problem.py}


\textbf{Explanation:}
\begin{itemize}
    \setlength\itemsep{0em}
    \item Am ales sa implementam iterativeDeepeningSearch, fiind un algoritm de cautare care genereaza solutia optima cu o abordare de tip DFS
\item la linia 20, in loc de a folosi o stiva data de noi, am folosit stiva de la apelul functiei, rezolvarea fiind recursiva% in line 4 we raise an NotDefinied error to make the debugging procces easier

\end{itemize}


\textbf{Commands:}
\begin{itemize}
    \setlength\itemsep{0em}
\item python pacman.py -l tinyMaze -p SearchAgent -a fn=ids
\item python pacman.py -l smallMaze -p SearchAgent -a fn=ids
    \item  python pacman.py -l mediumMaze -z .5 -p SearchAgent -a fn=ids
\item python pacman.py -l bigMaze -z .5 -p SearchAgent -a fn=ids% exemplu -l tinyMaze -p SearchAgent -a fn=dfs
        
\end{itemize}

\subsubsection{Questions}
This sub-section is dedicated to the additional questions that come along with the exercise. Please answer to the following questions:\newline


\subsubsection{Personal observations and notes}
Am urmat algoritmul de la curs.\newline
Pe layout-ul tinyMaze se expandeaza 94 de noduri. \newline
Pe layout-ul smallMaze se expandeaza 2295 noduri. \newline
Pe layout-ul mediumMaze se expandeaza 29857 noduri. \newline
Pe layout-ul bigMaze se expandeaza 119140 noduri. \newline \newline
Algoritmul de cautare este: 
\begin{itemize}
\item  complet daca spatiul starilor este finit
\end{itemize}
\begin{itemize}
\item  optim daca costul este constant intre noduri
\item are complexitatea timpului exponentiala 
\item are complexitatea spatiului liniara
\end{itemize}

% descrieti aici orice fel de probleme ati intampinat in timpul rezolvarii acestui task si modul cum in care le-ati solutionat

\vspace{0.75cm}

\subsection{References}

