\section{Uninformed search}

% ============================================= DFS ============================================= 
\subsection{Question 1 - Depth-first search}
% enuntul intrebarii
In this section the solution for the following problem will be presented: \newline


\textit{"In search.py, implement \textbf{Depth-First search(DFS) algorithm} in  function \textit{depthFirstSearch}. Don’t  forget that DFS graph search is graph-search with the frontier as a LIFO queue(Stack)."}.


\subsubsection{Code implementation}
This sub-section is dedicated to showcasing your own solution that you came up with for solving the above question. One has to put here any \textbf{code} that has been used for solving the above task, along with \textbf{comments} that explain every design decision made. To reference the code, please make use of the \textit{code lines number}. Additionally, complete this sub-section with any \textbf{command configurations} that you may have used during the implementation or testing process (please fill in \textit{just the arguments}). \\

\textbf{Code:}
% a se completa fisierul code/dfs.py
\inputminted[linenos]{python}{code/01_dfs.py}


\textbf{Explanation:}
\begin{itemize}
    \setlength\itemsep{0em}
    \item la linia 7 declaram frontiera ca o stiva
    \item la linia 18 si 48 ne folosim de un tip creat de noi, CustomNode, pentru a retine starea, actiunea, costul si parintele unui succesor
    \item la linia 19 se observa ca parintele pozitiei de start este (-1, -1)
    \item la linia 39 se observa ca solutia este construita pe parcursul algoritmului prin adaugarea actiunii nodului curent in lista solutiei
    \item la linia 71 se observa functia doesStackHaveThisItem pe care o apelam in acest algoritm
    % in line 4 we raise an NotDefinied error to make the debugging procces easier

\end{itemize}


\textbf{Commands:}
\begin{itemize}
    \setlength\itemsep{0em}
    \item python pacman.py -l tinyMaze -p SearchAgent
    \item python pacman.py -l smallMaze -p SearchAgent
    \item python pacman.py -l mediumMaze -p SearchAgent
    \item python pacman.py -l bigMaze -z .5 -p SearchAgent
      % exemplu -l tinyMaze -p SearchAgent -a fn=dfs
        
\end{itemize}

\subsubsection{Questions}
This sub-section is dedicated to the additional questions that come along with the exercise. Please answer to the following questions:\newline


\textbf{Q1:} Is the found solution optimal? Explain your answer.


\textbf{A1:} Nu este o solutie optima pentru rezolvarea problemei de cautare deoarece este un algoritm DFS, care e mai rapid decat alti algoritmi, dar nu garanteaza solutia optima. % scrieti raspunsul acestei intrebari si stergeti \newline, acesta ar trebui sa aiba maxim 100 de cuvinte


\textbf{Q2:} Run\textit{ autograder python autograder.py} and write the points for Question 1.


\textbf{A2:} Question q1: 4/4 

 % scrieti raspunsul acestei intrebari si stergeti \newline, acesta ar trebui sa aiba maxim 100 de cuvinte


\subsubsection{Personal observations and notes}
Pentru a verifica daca succesorul unui nod este deja in frontiera,
am creat o functie numita doesStackHaveThisItem (linia 74), care returneaza True daca a gasit nodul in frontiera si False daca nu l-a gasit.\newline
Pe layout-ul tinyMaze se expandeaza 16 de noduri. \newline
Pe layout-ul smallMaze se expandeaza 59 noduri. \newline
Pe layout-ul mediumMaze se expandeaza 146 noduri. \newline
Pe layout-ul bigMaze se expandeaza 391 noduri. \newline \newline
Algoritmul de cautare: 
\begin{itemize}
\item  nu este complet
\item  nu este optim
\item are complexitatea timpului exponentiala 
\item are complexitatea spatiului liniara
% descrieti aici orice fel de probleme ati intampinat in timpul rezolvarii acestui task si modul in care le-ati solutionat

\vspace{0.75cm}

% ============================================= BFS ============================================= 
\subsection{Question 2 - Breadth-first search}
% enuntul intrebarii
In this section the solution for the following problem will be presented:\newline


\textit{"In \textbf{search.py}, implement the \textbf{Breadth-First search} algorithm in function \textit{breadthFirstSearch}."}.


\subsubsection{Code implementation}
This sub-section is dedicated to showcasing your own solution that you came up with for solving the above question. One has to put here any \textbf{code} that has been used for solving the above task, along with \textbf{comments} that explain every design decision made. To reference the code, please make use of the \textit{code lines number}. Additionally, complete this sub-section with any \textbf{command configurations} that you may have used during the implementation or testing process (please fill in \textit{just the arguments}). \newline


\textbf{Code:}
% a se completa fisierul code/bfs.py
\inputminted[linenos]{python}{code/02_bfs.py}


\textbf{Explanation:}
\begin{itemize}
    \setlength\itemsep{0em}
    \item la linia 7 declaram frontiera ca o coada
    \item la linia 14 si 30 ne folosim de o structura de date de tip CurrentNode care retine starea, actiunea, costul si parintele unui nod
    \item la linia 32, deoarece nu am pus starea initiala in exploredList, verificam ca nodul succesor sa nu fie nodul de start
    \item la linia 52 se observa ca am parcurs exploredList in ordine inversa, pentru ca sa se realizeze comparatia dintre list si exploredList cat mai repede
    \item la linia 63 am declarat functia doesQueueHaveThisItem pe care o apelam in algoritm % in line 4 we raise an NotDefinied error to make the debugging procces easier

\end{itemize}


\textbf{Commands:}
\begin{itemize}
    \setlength\itemsep{0em}
    \item python pacman.py -l tinyMaze -p SearchAgent -a fn=bfs
    \item python pacman.py -l smallMaze -p SearchAgent -a fn=bfs
    \item python pacman.py -l mediumMaze -p SearchAgent -a
    fn=bfs
    \item python pacman.py -l bigMaze -z .5 -p SearchAgent -a fn=bfs %exemplu -l tinyMaze -p SearchAgent -a fn=dfs
        
\end{itemize}

\subsubsection{Questions}
This sub-section is dedicated to the additional questions that come along with the exercise. Please answer to the following questions:\newline


\textbf{Q1:} Is the found solution optimal? Explain your answer. 


\textbf{A1:} Da, solutia este optima pentru o problema de cautare, datorita modului de functionare a BFS-ului (cautarea in latime), cu toate ca expandeaza foarte multe noduri si este mai incet decat alti algoritmi de cautare. % scrieti raspunsul acestei intrebari si stergeti \newline, acesta ar trebui sa aiba maxim 100 de cuvinte


\textbf{Q2:} Run autograder \textit{python autograder.py} and write the points for Question 2.


\textbf{A2:} Question q2: 4/4 % scrieti raspunsul acestei intrebari si stergeti \newline, acesta ar trebui sa aiba maxim 100 de cuvinte


\subsubsection{Personal observations and notes}
Pentru a verifica daca un nod se afla in frontiera am creat o functie numita doesQueueHaveThisItem care returneaza True daca gaseste nodul in coada si False daca nu-l gaseste. \newline
Pentru a grabi gasirea solutiei, am ales sa punem in acelasi timp succesorii in frontiera si in lista de explorat. \newline
Pe layout-ul tinyMaze se expandeaza 15 de noduri. \newline
Pe layout-ul smallMaze se expandeaza 92 noduri. \newline
Pe layout-ul mediumMaze se expandeaza 269 noduri. \newline
Pe layout-ul bigMaze se expandeaza 620 noduri. \newline \newline
Algoritmul de cautare este: 
\begin{itemize}
\item complet daca spatiul starilor este finit
\item optim daca costul este constant intre noduri
\item are complexitatea timpului exponentiala 
\item are complexitatea spatiului exponentiala
% descrieti aici orice fel de probleme ati intampinat in timpul rezolvarii acestui task si modul cum in care le-ati solutionat

\vspace{0.75cm}


% ============================================= UCS ============================================= 
\subsection{Question 3 - Uniform-cost search}
% enuntul intrebarii
In this section the solution for the following problem will be presented:\newline

\textit{"In search.py,  implement  \textbf{Uniform-cost graph search} algorithm  in \textit{uniformCostSearchfunction}"}


\subsubsection{Code implementation}
This sub-section is dedicated to showcasing your own solution that you came up with for solving the above question. One has to put here any \textbf{code} that has been used for solving the above task, along with \textbf{comments} that explain every design decision made. To reference the code, please make use of the \textit{code lines number}. Additionally, complete this sub-section with any \textbf{command configurations} that you may have used during the implementation or testing process (please fill in \textit{just the arguments}). \newline


\textbf{Code:}
% a se completa fisierul code/ucs.py
\inputminted[linenos]{python}{code/03_ucs.py}



\textbf{Explanation:}
\begin{itemize}
    \setlength\itemsep{0em}
    \item % in line 4 we raise an NotDefinied error to make the debugging procces easier

\end{itemize}


\textbf{Commands:}
\begin{itemize}
    \setlength\itemsep{0em}
    \item  %exemplu -l tinyMaze -p SearchAgent -a fn=dfs
        
\end{itemize}

\subsubsection{Questions}
This sub-section is dedicated to the additional questions that come along with the exercise. Please answer to the following questions:\newline 

\textbf{Q1:} Compare the results to the ones obtained with DFS. Are the solutions different? Is the number of extended (explored) states smaller? Explain your answer. 

\textbf{A1:} \newline % scrieti raspunsul acestei intrebari inainte de \newline, acesta ar trebui sa aiba maxim 100 de cuvinte


\textbf{Q2:} Consider that some positions are more desirable than others. This can be modeled by a cost function which sets different values for the actions of stepping into positions. Identify in \textbf{searchAgents.py} the description of agents StayEastSearchAgent and StayWestSearchAgent and analyze the cost function. Why the cost .5 ** x for stepping into (x,y) is associated to StayWestAgen. 

\textbf{A2:} \newline % scrieti raspunsul acestei intrebari inainte de \newline, acesta ar trebui sa aiba maxim 100 de cuvinte


\textbf{Q3:} Run autograder \textit{python autograder.py} and write the points for Question 3.

\textbf{A3:} \newline % a se completa aici cu rezultatul din autograder

\subsubsection{Personal observations and notes}
% descrieti aici orice fel de probleme ati intampinat in timpul rezolvarii acestui task si modul cum in care le-ati solutionat
\vspace{0.75cm}

\subsection{References}
