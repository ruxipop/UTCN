 \section{Introducere}

 Implementarea proiectului se bazează pe logica propozitională a predicatelor. Logica propozițională se ocupă de propoziții (care pot fi false sau adevărate) și de relațiile dintre propoziții.
 
 Există 6 tipuri de simboluri în logica predicatelor:
 \begin{itemize}
    \setlength\itemsep{0em}
    \item Predicate: p, q, r...
    \item Constante: a, b, c, mașina...
    \item Variabile: x, y, z...
    \item Conective: |, ->, <->, -
    \item Paranteze: (,)
    \item Cuantificatori: forall, exists...
\end{itemize}


O propoziție atomică este una al cărei adevăr sau falsitate nu depinde de adevărul sau falsitatea oricărei alte propoziții.

   Următoarele propoziții sunt adevărate:
 \begin{itemize}
    \setlength\itemsep{0em}
    \item Simbolurile True și False  sunt propoziții atomice
    
    \item  Simbolurile P1, P2 etc. sunt propoziții atomice
    \item Dacă S este o propoziție => ¬S este o propoziție (negation)
    \item  Dacă S1 și S2 sunt propoziții => S1 &  S2  este o propoziție (conjunction)
    
    \item Dacă S1 și S2 sunt propoziții => S1 | S2 este o propoziție (disjunction)
    \item Dacă S1 și S2 sunt propoziții => S1 -> S2 este o propoziție (implication)
    \item Dacă S1 și S2 sunt propoziții => S1 <-> S2 este o propoziție (biconditional)
\end{itemize}

Tabele de adevăr:

1. AND                      
\\

\begin{left}
    \begin{tabular}{c |c| c}
       p  & q & p and q  \\
        \hline
       F  & F & F \\
        F  & T & F \\
         T  & F & F \\
          T  & T & T \\
    \end{tabular}
\end{left}
\\

2.OR
\\

\begin{left}
    \begin{tabular}{c |c| c}
       p  & q & p or q  \\
        \hline
       F  & F & F \\
        F  & T & T \\
         T  & F & T \\
          T  & T & T \\
    \end{tabular}
\end{left}
\\
\\

3.If ... then
\\

\begin{left}
    \begin{tabular}{c |c| c}
   
       p  & q & p -> q  \\
        \hline
       F  & F & T \\
        F  & T & T \\
         T  & F & F \\
          T  & T & T \\
    \end{tabular}
\end{left}
\\
\\

3.Iff
\\

\begin{left}
    \begin{tabular}{c |c| c}
       p  & q & p <-> q  \\
        \hline
       F  & F & T \\
        F  & T & F \\
         T  & F & F \\
          T  & T & T \\
    \end{tabular}
\end{left}