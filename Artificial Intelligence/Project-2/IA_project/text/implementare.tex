 \section{Implementare}

Prove9 este un tool folosit pentru a demonstra teoreme bazate pe logica propozițională și FOL(First Order Logic). Mace4 este folosit pentru a genera modele pe baza logicii propoziționale sau FOL.

Pentru a reprezenta în FOL o problemă bazată pe logica propozițională trebuie luat în considerare dimensiunea domeniului. Problema propusă de noi are un domeniu de dimensiune 4 (domain size: 4).
\newline
Constantele problemei sunt: M1,M2,M3,M4; reprezentând cele 4 mașini aflate în intersecție.
\newline
\subsection{Intersecție nedirijată}
Predicatele problemei pentru intersectia nedirijată  sunt:

 \begin{itemize}
    \setlength\itemsep{0em}
    \item semnalizeaza(x): mașina x semnalizează
    \item semnalizeaza st(x): mașina x semnalizează stânga
    \item semnalizeaza dr(x): mașina x semnalizează dreapta
    \item fata(x): mașina x merge în față
    \item dreapta(x): mașina x merge în dreapta
    \item stanga(x): mașina x merge in stânga
    \item in dreapta(x, y): mașina y se află în dreapta lui x
    \item prioritate1(x): mașina x are prima prioritate
    \item prioritate2(x): mașina x are prioritate dupa ce trece prima mașină
    \item prioritate3(x): mașina x are prioritate după ce trece prima și a doua mașină
    \item prioritate4(x): mașina x are prioritate după ce trece prima, a doua și a treia mașină
    \item  trece1(x): mașina x trece prima 
    \item  trece2(x): mașina x trece a doua
    \item trece3(x): mașina x trece a treia
    \item  trece4(x): mașina x trece a patra


\end{itemize}

\textbf{ Explicații:}

  \begin{itemize}
    \setlength\itemsep{0em}
    \item 	Deoarece într-o intersecție nedirijată se aplică prioritatea de dreapta, am ales să creez un predicat in dreapta(x, y) care o să arate care mașină se află în dreapta cărei mașini (de exemplu: in dreapta(M1, M2) înseamnă că mașina M2 este în dreapta lui M1).
    \item 	Dacă o mașină nu semnalizează, asta înseamnă că ea nu semnalizează la stânga sau la dreapta, adică merge în față.
    \item 	Predicatele prioritate1(x), prioritate2(x), prioritate3(x) și prioritate4(x) arată ordinea de prioritate a mașinilor. De exemplu, prioritate3(M4) ar însemna că M4 ar fi a treia mașină care ar avea prioritate după ce au trecut primele două mașini.
    \item Liniile 49-52 asigură că o singură mașină poate avea prioritate la un moment dat.
    \item Predicatele trece1(x), trece2(x), trece3(x) și trece4(x) indică ordinea de trecere a mașinilor. De exemplu, trece1(M2) ar însemna că mașina M2 trece prima în intersecție.
\end{itemize}



\textbf{Code:}
% a se completa fisierul code/dfs.py

    \inputminted[linenos]{C}{cod/aut.in}



    
\textbf{Comenzi:}

Pentru rularea codului ne-am folosit de tool-ul Prover9 care demonstrează dacă o mașină trece în ordine sau nu. Astfel, se pot pune următoarele goal-uri pentru a testa codul:


 \begin{itemize}
    \setlength\itemsep{0em}
    \item trece1(M2), trece2(M1), trece3(M4) sau trece4(M3) o să demonstreze ordinea trecerii mașinilor: M2 trece prima, M1 o să treacă a doua, M4 a treia și M3 ultima.
    \item trece1(M1), trece1(M3), trece1(M4), trece2(M2), trece2(M3), trece2(M4), trece3(M1), trece3(M2), trece3(M3), trece4(M1), trece4(M2) sau trece4(M4) vor da erori deoarece algoritmul nu o să reușească să găsească o demonstrație pentru ele.
    \item -trece1(M1), -trece1(M3), -trece1(M4), -trece2(M2), -trece2(M3), -trece2(M4), -trece3(M1), -trece3(M2), -trece3(M3), -trece4(M1), -trece4(M2) sau -trece4(M4) vor reuși să demonstreze faptul că mașinile nu vor trece în ordine în intersecție.
    \item -trece1(M2), -trece2(M1), -trece3(M4) sau -trece4(M3) vor da eroare deoarece algoritmul nu o să reușească să le demonstreze.

\end{itemize}


\textbf{Comanda pentru rularea codului din terminal:}
prover9 -f ”aut.in”

\newpage
\subsection{Intersecție dirijată}

Predicatele problemei pentru intersecția dirijată sunt:

\newline
 \begin{itemize}
    \setlength\itemsep{0em}
    \item  semnalizeaza(x): mașina x semnalizează
    \item semnalizeaza st(x): mașina x semnalizează stânga
    \item semnalizeaza  dr(x): mașina x semnalizează dreapta
    \item fata(x): mașina x merge în față

\item  dreapta(x): mașina x merge în dreapta
    \item stanga(x): mașina x merge în stânga
    \item semn(x, cedeaza), semn(x, prioritate), semn(x, stop): mașina x are semnul de cedează/prioritate/stop
    \item peAceeasiStradă(x, y, prioritate), peAceeasiStrada(x, y, faraPrioritate): mașinile x si y se află pe o stradă cu prioritate/fără prioritate
    \item trece1(x): mașina x trece prima
    \item trece2(x): mașina x trece a doua
    \item trece3(x): mașina x trece a treia
    \item trece4(x): mașina x trece a patra
\end{itemize}

\textbf{ Explicații:}
\newline
 \begin{itemize}
    \setlength\itemsep{0em}
	\item Predicatele semn(x, cedeaza), semn(x, prioritate) și semn(x, stop) arată semnul specific intrării în intersecție unei mașini (de exemplu, semn(M1, cedează) înseamnă că mașina are cedează trecerea).
\item La linia 43, dacă două mașini sunt pe drumul cu semnul de prioritate, atunci predicatul peAceeasiStrada(x, y, prioritate) este adevărat.
\item	La linia 46, dacă două mașini sunt pe drumul cu semnul  de cedează sau stop, atunci predicatul peAceeasiStrada(x, y, faraPrioritate) este adevărat.
\end{itemize}
\textbf{Code:}
% a se completa fisierul code/dfs.py

    \inputminted[linenos]{C}{cod/aut_dir.in}



    
\textbf{Comenzi:}

Pentru rularea codului ne-am folosit de tool-ul Prover9 care demonstrează dacă o mașină trece în ordine sau nu. Astfel, se pot pune următoarele goal-uri pentru a testa codul:


 \begin{itemize}
    \setlength\itemsep{0em}
    \item trece1(M2), trece2(M4), trece3(M3) sau trece4(M1) o să demonstreze ordinea trecerii mașinilor: M2 trece prima, M4 o să treacă a doua, M3 a treia și M1 ultima.
    \item trece1(M1), trece1(M3), trece1(M4), trece2(M1), trece2(M2), trece2(M3), trece3(M1), trece3(M2), trece3(M4), trece3(M2), trece3(M3) sau trece3(M4)  vor da erori deoarece algoritmul nu o să reușească să găsească o demonstrație pentru ele.
    \item -trece1(M1), -trece1(M3), -trece1(M4), -trece2(M1), -trece2(M2), -trece2(M3), -trece3(M1), -trece3(M2), -trece3(M4), -trece3(M2), -trece3(M3) sau -trece3(M4) vor reuși să demonstreze faptul că mașinile nu vor trece în ordine în intersecție.
    \item -trece1(M2), -trece2(M4), -trece3(M3) sau -trece4(M1) vor da eroare deoarece algoritmul nu o să reușească să le demonstreze.  

\end{itemize}


\textbf{Comanda pentru rularea codului din terminal:}
prover9 -f ”aut\_dir.in”
